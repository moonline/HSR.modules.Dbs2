%Pakete;
%A4, Report, 12pt
\documentclass[ngerman,a4paper,12pt]{scrreprt}
\usepackage[a4paper, right=20mm, left=20mm,top=20mm, bottom=30mm, marginparsep=5mm, marginparwidth=5mm, headheight=7mm, headsep=15mm,footskip=15mm]{geometry}

%Papierausrichtungen
\usepackage{pdflscape}
\usepackage{lscape}

%Deutsche Umlaute, Schriftart, Deutsche Bezeichnungen
\usepackage[utf8]{inputenc}
\usepackage[T1]{fontenc}
\usepackage[ngerman]{babel}

%quellcode
\usepackage{listings}

%tabellen
\usepackage{tabularx}

%listen und aufzählungen
\usepackage{paralist}

%farben
\usepackage[svgnames,table,hyperref]{xcolor}

%symbole
\usepackage{latexsym,textcomp}

%font
\usepackage{helvet}
\renewcommand{\familydefault}{\sfdefault}

%Abkürzungsverzeichnisse
\usepackage[printonlyused]{acronym}

%Bilder
\usepackage{graphicx} %Bilder
\usepackage{float}	  %"Floating" Objects, Bilder, Tabellen...
\usepackage[space]{grffile} %Leerzechen Problem bei includegraphics
\usepackage{wallpaper} %Seitenhintergrund setzen
\usepackage{transparent} %Transparenz

%for
\usepackage{forloop}
\usepackage{ifthen}

%Dokumenteigenschaften
\title{Repetitionsfragen Dbs2}
\author{Tobias Blaser}
\date{\today{}, Rapperswil}


%Kopf- /Fusszeile
\usepackage{fancyhdr}
\usepackage{lastpage}

\pagestyle{fancy}
	\fancyhf{} %alle Kopf- und Fußzeilenfelder bereinigen
	\renewcommand{\headrulewidth}{0pt} %obere Trennlinie
	\fancyfoot[L]{Seite \thepage/\pageref{LastPage}} %Fusszeile mitte
	\fancyfoot[R]{\today{}} %Fusszeile rechts
	\renewcommand{\footrulewidth}{0.4pt} %untere Trennlinie

%Kopf-/ Fusszeile auf chapter page
\fancypagestyle{plain} {
	\fancyhf{} %alle Kopf- und Fußzeilenfelder bereinigen
	\renewcommand{\headrulewidth}{0pt} %obere Trennlinie
	\fancyfoot[L]{Seite \thepage/\pageref{LastPage}} %Fusszeile mitte
	\fancyfoot[R]{\today{}} %Fusszeile rechts
	\renewcommand{\footrulewidth}{0.4pt} %untere Trennlinie
}

\usepackage{changepage}

% Abkürzungen für Kapitel, Titel und Listen
\input{commands/shortcutsListAndChapter}
\input{commands/TextStructuringBoxes}

%links, verlinktes Inhaltsverzeichnis, PDF Inhaltsverzeichnis
\usepackage[bookmarks=true,
bookmarksopen=true,
bookmarksnumbered=true,
breaklinks=true,
colorlinks=true,
linkcolor=black,
anchorcolor=black,
citecolor=black,
filecolor=black,
menucolor=black,
pagecolor=black,
urlcolor=black
]{hyperref} % Paket muss unbedingt als letzes eingebunden werden!

\usepackage{graphicx}
\begin{document}

% Inhaltsverzeichnis
\tableofcontents
\clearpage


\ch{Repetition Dbs1}
\ol
	\li Erklären Sie das ANSI 3-Ebenen Modell
	\li Erklären SIe OLTP
	\li Nennen Sie die vier ACID Kriterien. Wozu werden Sie verwendet?
	\li Lesen Sie mit einer group by Anfrage aus einer Liste mit Absolventen die Studiengangsnotendurchschnitte heraus.
\olS


\ch{Oracle PL/SQL}
\olR
	\li Was ist der Unterschied zwischen einer 'Stored Procedure' und einer Funktion?
	\li Wie ist eine Funktion aufgebaut, wie eine Stored Procedure? Wie führen Sie den Code jeweils aus (Einbetten in anderen Code?)?
	\li Wie funktioniert in PL/SQL das Exception Handling? Was sind unbenannte/benannte Benutzerexceptions, was benannte Systemexceptions?
	\li Für welche drei Anwendungszwecke setzen Sie 'Stored Procedures' ein?
	\li Welchen Vorteil bieten Stored Procedures in Zusammenhang mit Views?
	\li Wozu dient die Pseudotabelle 'DUAL'?
\olS


\ch{Stored Procedures}
\olR
	\li Erklären SIe den Unterschied zwischen einem anonymen PL/SQL Block und einer Stored Procedure. Nennen Sie je Vor- und Nachteile.
	\li Wie funktionieren Stored Procedures mit Java?
	\li Wie wird der Java Code verarbeitet und wo wird er gespeichert?
	\li Beschreiben Sie, was sie alles unternehmen müssen, um eine Java Stored Procedure zum Laufen zu bekommen.
	\li Machen Sie ein Beispiel, wie sie Stored Procedures mit Python verwenden.
\olS


\ch{Packages}
\olR
	\li Was sind PL/SQL Packages, wozu können Sie eingesetzt werden?
	\li Warum besitzt ein DBS keine gewöhliche Input/Output Shell? Wie können Sie trotzdem Angaben von einer Kommandozeile einlesen und ausgeben?
	\li Machen Sie eine Beispiel für ein Package
\olS


\ch{Cursors}
\olR
	\li Wozu verwenden Sie Cursors?
	\li Lesen Sie mittels eines Cursors eine Tabelle mit Standorten und Temperaturen aus. Ist die Temperatur höher als x Grad, so übertragen Sie die Datenbankzeile in eine neue Tabelle.
	Übergeben Sie x als Parameter, sodass der Cursor von aussen gesteuert werden kann.
	\li Was können Sie mit den vier Curso-Attributen machen?
\olS


\ch{Constraints}
\olR
	\li Was sind Constraints? Was können Sie damit machen?
	\li Was sind primäre und sekundäre Constraints, was starke und schwache Constraints?
	\li Auf welchen Objekten können Constraints definiert werden?
	\li Machen Sie je Beispiele:
		\ul
			\li Constraint anlegen
			\li Constraint löschen
			\li Aktivieren / Deaktivieren von Constraints
			\li Auflisten von Constraints
		\ulE
\olS


\ch{Triggers}
\se{Triggers}
\olR
	\li Wozu dienen Triggers? Nennen SIe vier Hauptanwendungen.
	\li Welche Ereignisse können einen Trigger auslösen?
	\li Was sind Before-Triggers, was After-Triggers? Was sind Row-Triggers, was Statementtriggers?
	\li Wozu dienen ':old' und ':new'?
	\li Mit welchen Rechten werden Triggers ausgeführt?
	\li Machen Sie ein Beispiel, wie mit Triggers die Daten überprüft werden können vor dem Einfügen.
	\li Machen Sie ein Beispiel für einen Insert Trigger, der abgeleitete Attribute berechnet.
	\li Erklären Sie, in welcher Reihenfolge und wann Triggers wie abgearbeitet werden.
	\li Was sind instead-of Triggers? Was logon-/logoff Triggers?
\olS

\se{Updateable Views}
\olR
	\li Wo liegt der Unterschied zwischen temporären Tabellen und Views?
	\li Warum können Views keine zusätzlichen Indexes erhalten?
	\li Was sind updateable Views? Warum können Views per Default nicht/nur mit sehr starken Einschränkungen updaten?
	\li Wie realisieren Sie updateable Views?
\olS

\se{Materialized Views}
\olR
	\li Was sind materialized Views?
	\li Was ist eine virtuelle Tabelle?
\olS


\ch{Datenstrukturen}
\se{Arrays}
\olR
	\li Was sind Arrays?
	\li Nennen Sie den Hauptunterschied zu Sets.
	\li Nennen Sie einige sinnvolle Anwendungen für die Verwendung von Arrays.
	\li Warum sollten Sie Array nur dann einsetzen, wenn Daten vorwiegend einmal geschrieben werden und nicht geändert werden?
	\li Wie legen definieren Sie ein Array als Spaltendatentyp? Wie schreiben Sie Daten in das Array? Wie lesen Sie Daten aus?
	\li Wozu verwenden Sie array\_to\_text() und unnest()?
	\li Wie funktionieren die Operatoren <@, = und \&\& ?
	\li Wie können Sie in einem Select Statement automatisch alle Arraywerte durchsuchen
\olS

\se{Graphen}
\olR
	\li Was ist ein Graph? Wozu können Sie ihn verwenden?
	\li Wie definieren Sie eine Spalte als Graph und fügen einen Graph ein? (Nicht im Unterricht behandel)
	\li Was ist ein CTE und wozu wird es verwendet? Was ist der Unterschied zur Subquery?
\olS

\se{Dictionaries}
\olR
	\li Was ist ein Dictionary? Wozu können Sie ihn verwenden?
	\li Wie definieren Sie eine Spalte als Dictionary? Wie fügen Sie Paare ein und wie lesen Sie welche aus?
	\li Machen Sie ein Beispiel: Legen Sie eine Tabelle (id,tags) an und fügen Sie einige Einträge mit jeweils einigen Tags ein. Lesen Sie alle Datensätze aus, die den Tag 'Ferien' enthalten.
\olS



\end{document}
